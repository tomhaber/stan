\documentclass[10pt]{report}

\usepackage[
scale=0.875,
stdmathitalics=true,
stdmathdigits=true]{lucimatx}
\linespread{1.02}
% \usepackage{helvet}

\usepackage{fancyvrb}
\usepackage{amsmath}
% \usepackage{amssymb}
\usepackage{graphicx}
\usepackage[usenames,dvipsnames,svgnames,table]{xcolor}

\setlength{\paperheight}{3in}
\setlength{\paperwidth}{4in}
\pdfpagewidth=\paperwidth
\pdfpageheight=\paperheight

\setlength{\textwidth}{3.75in}
\setlength{\textheight}{2.5in}

\setlength{\oddsidemargin}{-1.0in}
\setlength{\evensidemargin}{-1.0in}
\setlength{\topmargin}{-1.25in}

\newcommand{\sld}[1]{\newpage{\noindent\LARGE \ \ \
    \textcolor{MidnightBlue}{\bfseries #1}}\vspace*{4pt}}

\newcommand{\code}[1]{{\tt #1}}
\newcommand{\spc}{\hspace*{0.25in}}

\newcounter{gmlrx}
\newcounter{gmlry}
\newcommand{\gmnode}[3]{\put(#1,#2){\circle{20}}\put(#1,#2){\makebox(0,0){$#3$}}}
\newcommand{\gmplate}[5]{
  \setcounter{gmlrx}{#1}\addtocounter{gmlrx}{#3}
  \setcounter{gmlry}{#2}\addtocounter{gmlry}{-#4}
  \put(#1,#2){\line(1,0){#3}}
  \put(#1,#2){\line(0,-1){#4}}
  \put(\value{gmlrx},\value{gmlry}){\line(-1,0){#3}}
  \put(\value{gmlrx},\value{gmlry}){\line(0,1){#4}}
  \setcounter{gmlrx}{#1}\addtocounter{gmlrx}{5}
  \setcounter{gmlry}{#2}\addtocounter{gmlry}{-6}
  \put(\value{gmlrx},\value{gmlry}){\makebox(0,0){$#5$}}
}

\newcommand{\myemph}[1]{{\color{MidnightBlue}{\bfseries #1}}}
\newcommand{\mypart}[2]{{\newpage 
    \mbox{ }
    \vfill
    \noindent\spc\color{MidnightBlue}{\LARGE\bfseries #1\\[10pt]\spc\Huge{#2}}
    \vfill\vfill}
  \mbox{ }}

\begin{document}
\sf%
\vspace*{6pt}
% 
\noindent
\spc{\huge\bfseries \color{MidnightBlue}{Stan}}
\\[8pt]
\spc{\Large\bfseries \color{MidnightBlue}{Probabilistic Programming Language}}
\\
\vfill
\noindent
\spc\spc{\slshape\small Core Development Team:}
\\[2pt]
\spc\spc{\small Andrew Gelman, \  \myemph{Bob Carpenter}, \  Matt Hoffman,}
\\
\spc\spc{\small Daniel Lee, \  Ben Goodrich, \  Michael Betancourt, \ }
\\
\spc\spc{\small Marcus Brubaker, \   Jiqiang Guo, \  Peter Li, \ }
\\
\spc\spc{\small Allen Riddell, \  Marco Inacio, \ Jeffrey Arnold, \ }
\\
\spc\spc{\small Mitzi Morris, \ Rob Trangucci}
\vfill
\mbox{ } 
\hfill
\includegraphics[width=27.5pt]{img/stanlogo-main.pdf}\hspace*{9pt}
\\
\spc{\small Stan 2.5.0 \ \footnotesize (October 2014)}
\hfill 
{\footnotesize \tt mc-stan.org}

\sld{Why Stan?}
%
\begin{itemize}
\item {\slshape\bfseries Application}: Fit rich Bayesian statistical models
\end{itemize}
%
\vspace*{2pt}
\begin{itemize}
\item {\slshape Problem}: Gibbs and Metropolis too slow (diffusive) 
\item {\slshape Solution}: Hamiltonian Monte Carlo (flow)
%
\vspace*{8pt}
\item {\slshape Problem}:  Interpreters slow and unscalable
\item {\slshape Solution}: Compiled to C++
%
\vspace*{8pt}
\item {\slshape Problem}:  Need gradients of log posterior for HMC
\item {\slshape Solution}: Reverse-mode algorithmic differentation
\end{itemize}

\sld{Why? (cont.)}
%
\begin{itemize}
\item {\slshape Problem}:  Existing algo-diff slow, limited, unextensible
\item {\slshape Solution}: Our own algo-diff
%
\vspace*{8pt}
\item {\slshape Problem}:  Algo-diff requires functions templated on
  all args
\item {\slshape Solution}: Our own density library, Eigen linear
 algebra
%
\vspace*{8pt}
\item {\slshape Problem}:  Need unconstrained parameters for HMC
\item {\slshape Solution}: Variable transforms w. Jacobian determinants
%
\end{itemize}

\sld{Why? (cont.)}
%
\begin{itemize}
\item {\slshape Problem}:  Need ease of use of BUGS
\item {\slshape Solution}: Compile a domain-specific language
%
\vspace*{8pt}
\item {\slshape Problem}:  Pure directed graphical language inflexible
\item {\slshape Solution}: Imperative probabilistic programming
  language
\vspace*{8pt}
\item {\slshape Problem}:  Need to tune parameters for HMC
\item {\slshape Solution}: Tune step size and estimate mass matrix
  during warmup;  on-the-fly number of steps (NUTS)
%
\end{itemize}

\sld{Why? (cont.)}
\begin{itemize}
%
\vspace*{8pt}
\item {\slshape Problem}:  Efficient up-to-proportion density calcs
\item {\slshape Solution}: Density template metaprogramming 
%
\vspace*{8pt}
\item {\slshape Problem}:  Limited error checking, recovery
\item {\slshape Solution}: Static model typing, informative exceptions
%
\vspace*{8pt}
\item {\slshape Problem}:  Poor boundary behavior
\item {\slshape Solution}: Calculate limits (e.g. $\lim_{x \rightarrow
    0} x \log x$)
%
\end{itemize}

\sld{Why? (continued)}
\begin{itemize}
\item {\slshape Problem}:  Nobody knows everything
\item {\slshape Solution}: Expand project team with specialists
\vspace*{8pt}
\item {\slshape Problem}:  Expanding code and project team
\item {\slshape Solution}: GitHub: branch, pull 
  request, code review
\item {\slshape Solution}: Jenkins: continuous integration
\item {\slshape Solution}: ongoing refactoring and code simplification
%
\end{itemize}

\sld{Why? (continued)}
\begin{itemize}
\item {\slshape Problem}:  Heterogeneous user base
\item {\slshape Solution}: More interfaces (R, Python, MATLAB, Julia)
\item {\slshape Solution}:  domain-specific examples, tutorials
\vspace*{8pt}
\item {\slshape Problem}:  Restrictive licensing limits use
\item {\slshape Solution}: Code and doc open source
(BSD, CC-BY)
\end{itemize}

\sld{What is Stan?}

\begin{itemize}
\item Stan is an \myemph{imperative} probabilistic programming language
  \vspace*{-12pt}
  \begin{itemize}\footnotesize
  \item  cf., BUGS: declarative; \ Church: functional; \ Figaro: object-oriented
  \end{itemize}
\item Stan \myemph{program}
  \vspace*{-4pt}
  \begin{itemize}\small
  \item declares data and (constrained) parameter variables
  \item defines log posterior (or penalized likelihood)
  \end{itemize}
\item Stan \myemph{inference}
  \vspace*{-4pt}
  \begin{itemize}\small
  \item MCMC for full Bayesian inference
  \item MLE for point estimation
  \end{itemize}
\item Stan is \myemph{open source} \ {\footnotesize (BSD core C++, GPLv3 interfaces)}
  \\
  {\footnotesize hosted on GitHub; uses Eigen matrix lib, Boost C++ lib, googletest}
\end{itemize}

\sld{Who's Using Stan?}
\begin{itemize}
\item 850+ mailing list registrations;  5500 manual downloads (2.4.0)

\item Biological sciences: {\footnotesize
clinical drug trials, entomology, opthalmology,
neurology, genomics, agriculture, botany, fisheries,
cancer biology, epidemiology, population ecology, neurology
}
\item Physical sciences: {\footnotesize 
astrophysics, molecular biology, oceanography, climatology
}
\item Social sciences: {\footnotesize
 population dynamics, psycholinguistics, social networks, political science
}
\item Other: {\footnotesize materials engineering, finance, actuarial,
  sports, public health,
  collaborative filtering, educational testing}
\end{itemize}

\sld{Books and Model Sets}
% 
\begin{itemize}
\item 450+ page core language tutorial manual
\item Installation and getting started manuals by interface
  \\ {\footnotesize  (RStan, PyStan, CmdStan, MatlabStan, Stan.jl)}
\item BUGS and JAGS examples (most of all 3 volumes), 
\item Gelman and Hill, {\slshape Data Analysis Using Regression and
    Multilevel/Hierarchical Models}
\item Wagenmakers and Lee, {\slshape Bayesian Cognitive Modeling}
\item two other books in progress
\end{itemize}

\sld{Scaling and Evaluation}
\begin{itemize}
\item Types of scaling
  \vspace*{-4pt}
  \begin{itemize}\small
  \item more data
  \item more parameters
  \item \myemph{more complex models} \hfill (why we built Stan)
  \end{itemize}
\item for MCMC, measure \hfill (vs.\ BUGS / JAGS)
  \vspace*{-4pt}
  \begin{itemize}\small
  \item time to convergence \hfill (0.5--{\large$\infty$} faster)
  \item time per effective sample at mixing \hfill (ditto)
  \item memory usage \hfill (90\% less, linear scaling)
  \end{itemize}
\end{itemize}

\sld{Stan vs. JAGS (BUGS)}
\begin{itemize}
\item JAGS is the new BUGS, and former \myemph{state-of-the-art}, \myemph{black-box} MCMC method
  \\ {\footnotesize (Gibbs with slice sampling; chokes w.\ high posterior correlation)}
\item Stan (2.0) vs. BUGS on scaling multilevel 2PL-IRT model
  \vspace*{-4pt}
  \begin{itemize}\footnotesize
  \item Stan about \myemph{10 times faster}  ($n_{eff}/s$) and uses
    10\% memory
  \end{itemize}
  \vspace*{-3pt}
  \vfill
  \hspace*{-12pt}
  \includegraphics[height=0.6in]{img/irt_scale-eps-converted-to.pdf}
  \includegraphics[height=0.45in]{img/irt-memory-2.png}
\item Stan can handle problems that choke BUGS and JAGS
\end{itemize}

\mypart{Part I}{Stan Front End}

\sld{Platforms and Interfaces}

\begin{itemize}
\item \myemph{Platforms}
  \\
  Linux, Mac OS X, Windows
\item \myemph{C++ API}
  \\
  {\footnotesize portable, standards compliant (C++03 now, moving to C++11)}
\item \myemph{Interfaces}
  \vspace*{-4pt}
  \begin{itemize}\footnotesize
  \item \myemph{CmdStan}: Command-line or shell interface (direct executable)
  \item \myemph{RStan}: R interface (Rcpp in memory)
  \item \myemph{PyStan}: Python interface (Cython in memory)
  \item \myemph{MatlabStan$^*$}: MATLAB interface (lightweight external process)
  \item \myemph{Stan.jl$^*$}: Julia interface (lightweight external process)
  \end{itemize}
  {\footnotesize ${}^*$ User contributed}
\end{itemize}
\vfill

\sld{Example: \ Bernoulli}
\vfill
\spc\spc
\begin{minipage}[t]{0.8\textwidth}
  \begin{Verbatim}
    data {
      int<lower=0> N;
      int<lower=0,upper=1> y[N];
    }
    parameters {
      real<lower=0,upper=1> theta;
    } 
    model {
      y ~ bernoulli(theta);
    }
  \end{Verbatim}
\end{minipage}
\vfill
\vfill
\vfill
\mbox{ } \hfill {\small\slshape notes: {\tt theta} uniform on $[0,1]$, \ {\tt y} vectorized}
\vfill


\sld{RStan Execution}

\begin{minipage}[t]{\textwidth}
  \footnotesize
  \begin{Verbatim}
    > N <- 5;   y <- c(0,1,1,0,0);
    > fit <- stan("bernoulli.stan", data = c("N", "y"));
    > print(fit, digits=2)
  \end{Verbatim}
  \vspace*{1pt}
  \begin{Verbatim}[fontshape=sl]
    Inference for Stan model: bernoulli.
    4 chains, each with iter=2000; warmup=1000; thin=1; 

    mean  se_mean    sd   2.5%    50%  97.5%  n_eff  Rhat
    theta  0.43     0.01  0.18   0.11   0.42   0.78   1229     1
    lp__  -5.33     0.02  0.80  -7.46  -5.04  -4.78   1201     1
  \end{Verbatim}
  \vspace*{6pt}
  \begin{Verbatim}
    > hist( extract(fit)$theta )
  \end{Verbatim}
  \vspace*{-24pt}
  \hfill\includegraphics[width=1.25in,height=0.9in]{img/bernoulli-posterior-histo.pdf}
  \hspace*{24pt}
\end{minipage}


\sld{Basic Program Blocks}

\begin{itemize}
\item \myemph{\tt\bfseries data} \ (once) 
  \vspace*{-4pt}
  \begin{itemize}\small
  \item {\slshape content}: declare data types, sizes, and constraints
  \item {\slshape execute}: read from data source, validate constraints
  \end{itemize}
  % 
\item \myemph{\tt\bfseries parameters} \ (every log prob eval)
  \vspace*{-4pt}
  \begin{itemize}\small
  \item {\slshape content}: declare parameter types, sizes, and constraints
  \item {\slshape execute}: transform to constrained, Jacobian
  \end{itemize}
  % 
\item \myemph{\tt\bfseries model} \ (every log prob eval) 
  \vspace*{-4pt}
  \begin{itemize}\small
  \item {\slshape content}: statements definining posterior density
  \item {\slshape execute}: execute statements
  \end{itemize}
\end{itemize}

\sld{Derived Variable Blocks}

\begin{itemize}
\item \myemph{\tt\bfseries transformed data} (once after data)
  \vspace*{-4pt}
  \begin{itemize}\small
  \item {\slshape content}: declare and define transformed data variables
  \item {\slshape execute}: execute definition statements, validate constraints
  \end{itemize}
  % 
\item \myemph{\tt\bfseries transformed parameters} (every log prob eval)
  \vspace*{-4pt}
  \begin{itemize}\small
  \item {\slshape content}: declare and define transformed parameter vars
  \item {\slshape execute}: execute definition statements, validate constraints
  \end{itemize}
  % 
\item \myemph{\tt\bfseries generated quantities} (once per draw, 
  \code{double} type)
  \vspace*{-4pt}
  \begin{itemize}\small
  \item {\slshape content}: declare and define generated quantity
    variables; \\
    includes pseudo-random number generators
    \\
    {\footnotesize (for posterior predictions, event probabilities,
      decision making)}
  \item {\slshape execute}: execute definition statements, validate constraints
  \end{itemize}
  % 
\end{itemize}


\sld{User-Defined Functions (Stan 2.3)}

\begin{itemize}
\item \myemph{\tt\bfseries functions} \ (compiled with model)
  \vspace*{-4pt}
  \begin{itemize}\small
  \item {\slshape content}: declare and define general (recursive) functions
    \\
    {\small (use them elsewhere in program)}
  \item {\slshape execute}: compile with model
  \end{itemize}
  \vspace*{6pt}
\item Example
  \\[6pt]
  \begin{minipage}[t]{0.8\textwidth}
    \footnotesize
    \begin{Verbatim}
      functions {

        real relative_difference(real u, real v) {
          return 2 * fabs(u - v) / (fabs(u) + fabs(v));
        }

      }
    \end{Verbatim}
  \end{minipage}
\end{itemize}

\sld{Variable and Expression Types}
\\[3pt]
\hspace*{17pt}Variables and expressions are \myemph{strongly, statically typed}.
\begin{itemize}
\item \myemph{Primitive}: {\tt\small int}, \ {\tt\small real}
\item \myemph{Matrix}: {\tt\small matrix[M,N]}, \ {\tt\small vector[M]}, \ {\tt\small row\_vector[N]}
\item \myemph{Bounded}: primitive or matrix, with 
  \\ {\tt\small <lower=L>}, \ {\tt\small <upper=U>}, \ {\tt\small <lower=L,upper=U>}
\item \myemph{Constrained Vectors}: {\tt\small simplex[K]}, \ {\tt\small
    ordered[N]},
  \\ {\tt\small positive\_ordered[N]}, \ {\tt\small unit\_length[N]}
\item \myemph{Constrained Matrices}: {\tt\small cov\_matrix[K]}, \ {\tt\small
    corr\_matrix[K]}, \ {\tt\small cholesky\_factor\_cov[M,N]}, \
  {\tt\small cholesky\_factor\_corr[K]}
\item \myemph{Arrays:}  of any type (and dimensionality)
\end{itemize}

\sld{Logical Operators}
\vfill
\noindent\spc
{\footnotesize
  \begin{tabular}{c|ccl|l}
    {\it Op.} & {\it Prec.} & {\it Assoc.} & {\it
      Placement} & {\it Description}
    \\ \hline \hline
    \code{||} & 9 & left & binary infix & logical or
    \\ \hline
    \Verb|&&| & 8 & left & binary infix & logical and
    \\ \hline
    \Verb|==| & 7 & left & binary infix & equality
    \\
    \Verb|!=| & 7 & left & binary infix & inequality
    \\ \hline
    \Verb|<| & 6 & left & binary infix & less than
    \\
    \Verb|<=| & 6 & left & binary infix & less than or equal
    \\
    \Verb|>| & 6 & left & binary infix & greater than 
    \\
    \Verb|>=| & 6 & left & binary infix & greater than or equal
  \end{tabular}
}
\vfill
\vfill

\sld{Arithmetic and Matrix Operators}
\vfill
\noindent\spc
{\footnotesize
  \begin{tabular}{c|ccl|l}
    {\it Op.} & {\it Prec.} & {\it Assoc.} & {\it
      Placement} & {\it Description}
    \\ \hline \hline

    \code{+} & 5 & left & binary infix & addition
    \\
    \code{-} & 5 & left & binary infix & subtraction
    \\ \hline
    \code{*} & 4 & left & binary infix & multiplication
    \\
    \code{/} & 4 & left & binary infix & (right) division
    \\ \hline
    \Verb|\| & 3 & left & binary infix & left division
    \\ \hline
    \code{.*} & 2 & left & binary infix & elementwise multiplication
    \\
    \code{./} & 2 & left & binary infix & elementwise division
    \\ \hline
    \code{!} & 1 & n/a & unary prefix & logical negation
    \\
    \code{-} & 1 & n/a & unary prefix & negation
    \\ 
    \code{+} & 1 & n/a & unary prefix & promotion (no-op in Stan)
    \\ \hline
    \Verb|^| & 2 & right & binary infix & exponentiation
    \\ \hline
    \code{'} & 0 & n/a & unary postfix & transposition
    \\ \hline \hline
    \code{()} & 0 & n/a & prefix, wrap & function application
    \\
    \code{[]} & 0 & left & prefix, wrap & array, matrix indexing
  \end{tabular}
}
\vfill

\sld{Built-in Math Functions}

\begin{itemize}
\item All built-in \myemph{C++ functions and operators}
  \\
  {\footnotesize C math, TR1, C++11, including all trig, pow, and
    special log1m, erf, erfc, fma, atan2, etc.}
\item Extensive library of \myemph{statistical functions}
  \\
  {\footnotesize e.g., softmax,
    log gamma and digamma functions, beta functions, Bessel functions of
    and second kind, etc.}
  % 
\item Efficient, arithmetically stable \myemph{compound functions}
  \\
  {\footnotesize e.g., multiply log, log sum of
    exponentials, log inverse logit}
\end{itemize}

\sld{Built-in Matrix Functions}

\begin{itemize}
\item \myemph{Basic arithmetic}: all arithmetic operators
\item \myemph{Elementwise arithmetic}: vectorized operations
\item \myemph{Solvers}: matrix division, (log) determinant,
  inverse 
\item \myemph{Decompositions}: QR, Eigenvalues and Eigenvectors, 
  \\
  Cholesky factorization, singular value decomposition
\item \myemph{Compound Operations}: quadratic forms, variance scaling
\item \myemph{Ordering, Slicing, Broadcasting}: sort, rank, block, rep
\item \myemph{Reductions}: sum, product, norms
\item \myemph{Specializations}: triangular, positive-definite, etc.
\end{itemize}

\sld{Differential Equation Solver}
\begin{itemize}
\item Auto-diff solutions w.r.t.\ parameters
\item Integrate coupled system for solution with partials
\item Auto-diff coupled Jacobian for stiff systems
  \vfill
\item C++ prototype integrated for large PK/PD models
  \vspace*{-4pt}
  \begin{itemize}\footnotesize
  \item 
    Project with Novartis: 
    longitudinal clinical trial w.\ multiple drugs, dosings, placebo control,
    hierarchical model of patient-level effects, meta-analysis
  \item
    Collaborators: Frederic Bois, Amy Racine, Sebastian Weber
  \end{itemize}
\end{itemize}

\sld{Distribution Library}

\begin{itemize}
\item Each distribution has
  \vspace*{-4pt}
  \begin{itemize}\small
  \item log density or mass function
  \item cumulative distribution functions, plus complementary versions,
    plus log scale
  \item pseudo Random number generators
  \end{itemize}
\item Alternative parameterizations
  \\
  {\footnotesize (e.g., Cholesky-based multi-normal,
    log-scale Poisson, logit-scale Bernoulli)}
\item New multivariate correlation matrix density: LKJ
  \\
  {\footnotesize degrees of freedom controls 
    shrinkage to (expansion from) unit matrix}
\end{itemize}

\sld{Statements}

\begin{itemize}
\item \myemph{Sampling}: \ {\footnotesize \Verb|y ~ normal(mu,sigma)|}
  \ \ \ {\footnotesize (increments log probability)}
  % 
\item \myemph{Log probability}: \ {\footnotesize increment\_log\_prob(lp);}
\item \myemph{Assignment}: \  {\footnotesize \code{y\_hat <- x * beta};}
  % 
\item \myemph{For loop}: \ {\footnotesize \code{for (n in 1:N) ...}}
  % 
\item \myemph{While loop}: \ {\footnotesize \code{while (cond) ...}}
  % 
\item \myemph{Conditional}: \ {\footnotesize
    \code{if (cond) ...; else if (cond) ...;  else ...;}}
\item \myemph{Block}: \ {\footnotesize \Verb|{ ... }|}  \ \ \ {\footnotesize
    (allows local variables)}
\item \myemph{Print}: \ {\footnotesize print("theta=",theta);}
\end{itemize}

\sld{Full Bayes with MCMC}

\begin{itemize}
\item Adaptive \myemph{Hamiltonian Monte Carlo} (HMC)
  % 
\item Adaptation \myemph{during warmup}
  \vspace*{-4pt}
  \begin{itemize}\small
  \item step size adapted to target Metropolis acceptance rate
  \item mass matrix estimated with regularization
    {\footnotesize
      \\ sample covariance of second half of warmup iterations
      \\ (assumes constant posterior curvature)
    }
  \end{itemize}
  % 
\item Adaptation \myemph{during sampling}
  \vspace*{-4pt}
  \begin{itemize}\small
  \item number of steps
    \\
    {\footnotesize aka no-U-turn sampler (NUTS)}
  \end{itemize}
  % 
\item \myemph{Initialization} user-specified or random unconstrained
\end{itemize}


\sld{Posterior Inference}

\begin{itemize}
\item Generated quantities block for \myemph{inference}
  \\ {\footnotesize (predictions, decisions, and event probabilities)}
\item \myemph{Extractors} for samples in RStan and PyStan
\item Coda-like \myemph{posterior summary}
  \vspace*{-4pt}
  \begin{itemize}\small
  \item posterior mean w.\ standard error, standard deviation, quantiles
  \item split-$\hat{R}$ multi-chain convergence diagnostic (Gelman and ...)
  \item multi-chain effective sample size estimation (FFT algorithm)
  \end{itemize}
\item Model comparison with \myemph{WAIC}
  \\
  {\footnotesize (internal log likelihoods; external cross-sample statistics)}
\end{itemize}

\sld{Penalized MLE}

\begin{itemize}
\item Posterior \myemph{mode finding} via BFGS optimization
  \\ {\footnotesize (uses model gradient, efficiently approximates Hessian)}
\item \myemph{Disables Jacobians} for parameter inverse transforms
\item Models, data, initialization as in MCMC
  \vfill
\item \myemph{Very Near Future}
  \vspace*{-4pt}
  \begin{itemize}\small
  \item  \myemph{Standard errors} on unconstrained scale
    \\
    {\footnotesize  (estimated using curvature of penalized log likelihood function}
  \item Standard errors \myemph{on constrained scale})
    \\
    {\footnotesize  (sample unconstrained approximation and inverse transform)}
  \item L-BFGS optimizer
  \end{itemize}      
\end{itemize}

\sld{Stan as a Research Tool}

\begin{itemize}
\item Stan can be used to \myemph{explore algorithms}
\item Models transformed to \myemph{unconstrained support} on $\mathbb{R}^n$
\item Once a model is compiled, have
  \vspace*{-4pt}
  \begin{itemize}\small
  \item \myemph{log probability, gradient, and Hessian}
  \item data I/O and parameter initialization
  \item model provides variable names and dimensionalities
  \item transforms to and from constrained representation 
    \\ {\footnotesize (with or without Jacobian)}
  \end{itemize}
\item {Very Near Future:}
  \vspace*{-4pt}
  \begin{itemize}\small
  \item second- and \myemph{higher-order derivatives} via auto-diff
  \end{itemize}
\end{itemize}


\mypart{Part II}{Under the Hood}

\sld{Euclidean Hamiltonian}

\begin{itemize}
\item \myemph{Phase space}: $q$ position (parameters); \ $p$ momentum
\item \myemph{Posterior density}: $\pi(q)$
\item \myemph{Mass matrix}: $M$
\item \myemph{Potential energy}: $V(q) = -\log \pi(q)$
\item \myemph{Kinetic energy}: $T(p) = \frac{1}{2} p^{\top} M^{-1} p$
\item \myemph{Hamiltonian}:  $H(p,q) = V(q) + T(p)$
\item \myemph{Diff eqs}:
  \[
  \frac{dq}{dt} \ = \  + \frac{\partial H}{\partial p}
  \hspace*{48pt}
  \frac{dp}{dt} \ = \ - \frac{\partial H}{\partial q}
  \]
\end{itemize}

\sld{Leapfrog Integrator Steps}
\begin{itemize}
\item Solves Hamilton's equations by \myemph{simulating dynamics}
  \\
  {\footnotesize (symplectic [volume preserving]; $\epsilon^3$ error per step, $\epsilon^2$ total error)}
\item Given: \myemph{step size} $\epsilon$, \myemph{mass matrix} $M$, \myemph{parameters} $q$
\item \myemph{Initialize kinetic} energy, $p \sim {\sf
    Normal}(0,\mbox{\bf I})$
\item \myemph{Repeat} for $L$ leapfrog steps:
  \begin{eqnarray*}
    p & \leftarrow &
    p - \frac{\epsilon}{2} \, \frac{\partial V(q)}{\partial q}
    \ \ \ \ \ \ \mbox{[half step in momentum]}
    \\[6pt]
    q & \leftarrow &
    q + \epsilon \, M^{-1} \, p
    \ \ \ \ \ \ \ \  \mbox{[full step in position]}
    \\[6pt]
    p & \leftarrow &
    p - \frac{\epsilon}{2} \, \frac{\partial V(q)}{\partial q}
    \ \ \ \ \ \ \mbox{[half step in momentum]}
  \end{eqnarray*}
\end{itemize}

\sld{Standard HMC}
\begin{itemize}
\item \myemph{Initialize parameters} diffusely
  \\ {\footnotesize Stan's default: \ $q \sim \mbox{\sf
      Uniform}(-2,2)$ on unconstrained scale}
\item For each draw
  \vspace*{-3pt}
  \begin{itemize}\small
  \item leapfrog integrator \myemph{generates proposal}
  \item \myemph{Metropolis accept} step ensures \myemph{detailed balance}
  \end{itemize}
  \vfill
\item \myemph{Balancing act}: small $\epsilon$ has low error, requires many steps
\item Results \myemph{highly sensitive} to step size $\epsilon$ and mass matrix
  $M$
\end{itemize}

\sld{Tuning HMC During Warmup}

\begin{itemize}
\item \myemph{Chicken-and-egg} problem
  \vspace*{-4pt}
  \begin{itemize}\small
  \item convergence to high mass volume requires adaptation 
  \item adaptation requires convergence
  \end{itemize}
\item During warmup, tune
  \vspace*{-4pt}
  \begin{itemize}\small
  \item \myemph{step size}: line search to achieve target acceptance
    rate
  \item \myemph{mass matrix}: estimate with second half of warmup
  \end{itemize}
\item Use exponentially growing adaptation block sizes
\end{itemize}

\sld{Position-Independent Curvature}

\begin{itemize}
\item \myemph{Euclidean} HMC uses \myemph{global mass matrix} $M$
\item Works for densities with \myemph{position-independent curvature}
\item \myemph{Counterexample}: hierarchical model
  \vspace*{-4pt}
  \begin{itemize}\small
  \item hierarchical variance parameter controls lower-level scale
  \item mitigate by reducing target acceptance rate
  \end{itemize}
  \vfill
\item \myemph{Riemannian-manifold} HMC (coming soon)
  \vspace*{-4pt}
  \begin{itemize}\small
  \item automatically adapts to varying curvature
  \item no need to estimate mass matrix
  \item need to regularize Hessian-based curvature estimate
    \\ {\footnotesize (Betancourt {\slshape arXiv}; SoftAbs metric)}
  \end{itemize}
\end{itemize}

\sld{Adapting HMC During Sampling}

\begin{itemize}
\item No-U-turn sampler (\myemph{NUTS})
\item Subtle algorithm to maintain \myemph{detailed balance}
\item Move randomly \myemph{forward or backward in time}
\item Double number of leapfrog steps each move (\myemph{binary tree})
\item Stop when a subtree makes a \myemph{U-turn}
  \\ {\footnotesize (rare: throw away second half if not end to end U-turn)}
\item \myemph{Slice sample} points along last branch of tree
\item \myemph{Generalized} to Riemannian-manifold HMC
  \\ {\footnotesize (Betancourt, arXiv)}
\end{itemize}

\sld{NUTS vs.\ Gibbs and Metropolis}

\includegraphics[width=0.9\textwidth]{img/nuts-vs.pdf}
\begin{itemize}
\item Two dimensions of highly correlated 250-dim distribution
\item 1M samples from Metropolis, 1M from Gibbs (thin to 1K)
\item 1K samples from NUTS, 1K independent draws
\end{itemize}

\sld{NUTS vs.\ Basic HMC}

\includegraphics[width=0.9\textwidth]{img/nuts-ess-1.pdf}
{\small
  \begin{itemize}
  \item 250-D normal and logistic regression models
  \item Vertical axis is effective sample size per sample (bigger better)
  \item Left) NUTS; \ \ Right) HMC with increasing $t = \epsilon L$
  \end{itemize}
}

\sld{NUTS vs.\ Basic HMC II}

\includegraphics[width=0.9\textwidth]{img/nuts-ess-2.pdf}

{\small
  \begin{itemize}
  \item Hierarchical logistic regression and stochastic volatility
  \item Simulation time $t$ is $\epsilon \ L$, step size ($\epsilon$)
    times number of steps ($L$)
  \item NUTS can beat optimally tuned HMC (latter very expensive)
  \end{itemize}
}


\sld{Reverse-Mode Auto Diff}
\begin{itemize}
\item Eval gradient in small multiple of function eval time
  \\
  {\footnotesize (independent of dimensionality)}
\item Templated \myemph{C++ overload} for all functions
\item Code \myemph{partial derivatives} for basic operations
\item Function evaluation builds up \myemph{expression tree}
\item Dynamic program propagates \myemph{chain rule} in reverse pass
\item Extensible w.\ \myemph{object-oriented} custom partial propagation
\item Arena-based \myemph{memory management}
  \\ {\footnotesize (customize \code{operator new})}
\end{itemize}

\sld{Forward-Mode Auto Diff}
\begin{itemize}
\item Templated \myemph{C++ overload} for all functions
\item Code \myemph{partial derivatives} for basic operations
\item Function evaluation propagates \myemph{chain rule} forward
\item Nest reverse-mode in forward for \myemph{higher-order}
\item \myemph{Jacobians}
  \vspace*{-4pt}
  \begin{itemize}\small
  \item Rerun propagation pass in reverse mode
  \item Rerun forward construction with forward mode
  \end{itemize}
  \vfill
\item Faster autodiff rewrite coming in six months to one year
\end{itemize}

\sld{Autodiff Functionals}

\begin{itemize}
\item Fully encapsulates autodiff in C++
\item Autodiff operations are functionals {\footnotesize (higher-order functions)}
  \begin{itemize}\small
  \item gradients, Jacobians, gradient-vector product
  \item directional derivative
  \item Hessian-vector product
  \item Hessian
  \item gradient of trace of matrix-Hessian product
    \\ {\footnotesize (for SoftAbs RHMC)}
  \end{itemize}
\item Functions to differentiate coded as functors (or pointers)
  \\ {\footnotesize (enables dynamic C++ bind or lambda)}
\end{itemize}

\sld{Variable Transforms}
\begin{itemize}
\item Code HMC and optimization with $\mathbb{R}^n$ \myemph{support}
\item Transform constrained parameters to unconstrained
  \vspace*{-2pt}
  {\small
    \begin{itemize}
    \item lower (upper) bound: offset (negated) log transform
    \item lower and upper bound: scaled, offset logit transform
    \item simplex: centered, stick-breaking logit transform
    \item ordered: free first element, log transform offsets
    \item unit length: spherical coordinates
    \item covariance matrix: Cholesky factor positive diagonal 
    \item correlation matrix: rows unit length via quadratic stick-breaking
    \end{itemize}
  }
\end{itemize}


\sld{Variable Transforms (cont.)}
\begin{itemize}
\item Inverse transform from unconstrained $\mathbb{R}^n$
\item Evaluate log probability in model block on natural scale
\item Optionally adjust log probability for change of variables
  \\ {\footnotesize (add log determinant of inverse transform Jacobian)}
\end{itemize}

\sld{Parsing and Compilation}
\begin{itemize}
\item Stan code \myemph{parsed} to abstract syntax tree (AST)
  \\ {\footnotesize (Boost Spirit Qi, recursive descent, lazy semantic
    actions)}
\item C++ model class \myemph{code generation} from AST
  \\ {\footnotesize (Boost Variant)}
\item C++ code \myemph{compilation}
\item \myemph{Dynamic linking} for RStan, PyStan
\end{itemize}

\sld{Coding Probability Functions}
\begin{itemize}
\item \myemph{Vectorized} to allow scalar or container arguments
  \\ {\footnotesize (containers all same shape; scalars broadcast as necessary)}
\item Avoid \myemph{repeated computations}, e.g. $\log \sigma$ in
  \hspace*{-18pt}
  {\small
    \begin{eqnarray*}
      \textstyle \log \, \mbox{\sf Normal}(y | \mu, \sigma)
      & = & \textstyle \sum_{n=1}^N \log \, \mbox{\sf Normal}(y_n | \mu,\sigma)
      \\[4pt]
      & = & \textstyle \sum_{n=1}^N  - \log \sqrt{2\pi} \ - \log \sigma \ -
      \frac{\textstyle y_n - \mu}{\textstyle 2\sigma^2}
    \end{eqnarray*}
  }
\item recursive \myemph{expression templates} to broadcast and cache scalars,
  generalize containers (arrays, matrices, vectors)
\item \myemph{traits} metaprogram to \myemph{drop constants} (e.g., $-\log
  \sqrt{2 \pi}$ or $\log \sigma$ if constant) 
  and calculate intermediate and return types
\end{itemize}

\sld{Models with Discrete Parameters}
\begin{itemize}
\item e.g., simple mixture models, survival models, HMMs,
  discrete measurement error models, missing data
\item \myemph{Marginalize out} discrete parameters
\item Efficient sampling due to \myemph{Rao-Blackwellization}
\item Inference straightforward with expectations
  \vspace*{12pt}
\item Too \myemph{difficult} for many of our users
  \\
  {\small (exploring encapsulation options)}
\end{itemize}

\sld{Models with Missing Data}
\begin{itemize}
\item In principle, missing data just \myemph{additional parameters}
\item In practice, how to declare? 
  \begin{itemize}
  \item \myemph{observed} data as data variables
  \item \myemph{missing} data as parameters
  \item combine into single vector 
    \\ {\footnotesize (in transformed parameters or local in model)}
  \end{itemize}
\end{itemize}




\mypart{Part III}{What's Next?}

\sld{Higher-Order Auto-diff}
\begin{itemize}
\item Finish higher-order auto-diff for probability functions
\item May punt some cumulative distribution functions
  \\
  {\footnotesize (Black art iterative algorithms required)}
  \vfill
\item Code complete; under testing
\end{itemize}

\sld{Riemannian Manifold HMC}
\begin{itemize}
\item \myemph{NUTS} generalized to RHMC
  \\ {\footnotesize (Betancourt {\slshape arXiv} paper)}
\item \myemph{SoftAbs} metric
  \vspace*{-4pt}
  \begin{itemize}\footnotesize
  \item Eigendecompose Hessian
  \item \myemph{positive definite} with positive eigenvalues
  \item \myemph{condition} by narrowing eigenvalue range
  \item Betancourt {\slshape arXiv}\ paper
  \end{itemize}
  \vfill
\item Code complete; awaiting higher-order auto-diff
\end{itemize}

\sld{Adiabatic Sampling}
\begin{itemize}
\item Physically motivated alternative to ``simulated''
  \myemph{annealing and tempering} (not really simulated!)
\item Supplies external \myemph{heat bath}
\item Operates through \myemph{contact manifold}
\item System relaxes more naturally between energy levels
  \vfill
\item Prototype complete
  \\ {\small (Betancourt paper on {\slshape arXiv}; for geometers)}
\end{itemize}

\sld{Maximum Marginal Likelihood}
\begin{itemize}
\item Fast, Approximate Inference
\item Marginalize out lower-level parameters
\item Optimize higher-level parameters and fix
\item Optimize lower-level parameters given higher-level
\item Errors estimated as in MLE
  \vfill
\item Design complete; awaiting parameter tagging
\end{itemize}

\sld{``Black Box'' VB}
\begin{itemize}
\item Fast, Approximate Inference
\item \myemph{Black box} so can run any model
  \\
  {\footnotesize (Laplace or other approximations)}
\item Stochastic, data-streaming \myemph{variational Bayes} (VB)
\item Optimize parameteric approximation to posterior to minimize KL
  divergence 
  \vfill
\item Prototype stage (collaborating with Alp Kucukelbir, Dave Blei,
  Rajesh Ranganath)
\end{itemize}

\sld{``Black Box'' EP}
\begin{itemize}
\item Fast, Approximate Inference
\item Data-parallel \myemph{expectation propagation} (EP)
  \\ {\footnotesize (cavity distributions provide general shard combination)}
\item Optimize parameteric approximation to posterior to minimize KL
  divergence 
  {\footnotesize (VB, EP measure divergence in opposite directions)}
  \vfill
  \vfill
\item Design stage (collaborating with Nicolas Chopin, Christian
  Robert, John Cunningham, Aki Vehtari, Pasi Jyl\"anki)
\end{itemize}

\mypart{}{The End}

\sld{Stan's Namesake}
\begin{itemize}
\item Stanislaw Ulam (1909--1984)
\item Co-inventor of Monte Carlo method (and hydrogen bomb)
\item[]
  \begin{center}
    \includegraphics[width=0.25\textwidth]{img/ulam-fermiac.jpg}
  \end{center}
  {\small
  \item Ulam holding the Fermiac, Enrico Fermi's physical Monte Carlo simulator
    for random neutron diffusion}
\end{itemize}



\mypart{Appendix I}{Bayesian Data Analysis}

\sld{Bayesian Data Analysis}
\begin{itemize}
\item ``By {Bayesian data analysis}, we mean {practical methods}
  for making {inferences} from {data} using {probability models}
  for quantities we {observe} and about which we {wish to learn}.''
  % 
\item ``The essential characteristic of Bayesian methods is
  their \myemph{explict use of probability for quantifying uncertainty}
  in inferences based on statistical analysis.''
\end{itemize}
% 
\vfill\hfill{\footnotesize Gelman et al., {\slshape Bayesian Data Analysis},
  3rd edition, 2013}


\sld{Bayesian Mechanics}
% 
\begin{enumerate}
\item Set up full probability model 
  \vspace*{-4pt}
  \begin{itemize}
  \item for all observable \& unobservable quantities
  \item consistent w. problem knowledge \& data collection
  \end{itemize}
  % 
\item Condition on observed data
  \vspace*{-4pt}
  \begin{itemize}
  \item caclulate posterior probability of unobserved quantities
    conditional on observed quantities
  \end{itemize}
  % 
\item Evaluate 
  \vspace*{-4pt}
  \begin{itemize}
  \item model fit 
  \item implications of posterior
  \end{itemize}
\end{enumerate}

\vfill\hfill {\footnotesize {\slshape Ibid.}}

\sld{Basic Quantities}
% 
\begin{itemize}
\item Basic Quantities
  \vspace*{-4pt}
  \begin{itemize}
  \item $y$: \ observed data
  \item $\tilde{y}$: \ unknown, potentially observable quantities
  \item $\theta$: \ parameters (and other unobserved quantities)
  \item $x$: \ constants, predictors for conditional models
  \end{itemize}
\item Random models for things that could've been otherwise
  \begin{itemize}
  \item Everyone: Model data $y$ as random
  \item Bayesians:  Model parameters $\theta$ as random
  \end{itemize}
\end{itemize}


\sld{Distribution Naming Conventions}

\begin{itemize}
\item \myemph{Joint}: \ $p(y,\theta)$
\item \myemph{Sampling / Likelihood}: \ $p(y|\theta)$
\item \myemph{Prior}: \ $p(\theta)$
\item \myemph{Posterior}: \ $p(\theta|y)$
\item \myemph{Data Marginal}: \ $p(y)$
\item \myemph{Posterior Predictive}: \ $p(\tilde{y}|y)$
\end{itemize}

\noindent
\spc
{\footnotesize $y$ modeled data, \, $\theta$ parameters, \, $\tilde{y}$ predictions,}
\\[4pt]
\spc
{\footnotesize implicit: $x, \tilde{x}$ unmodeled data (for $y$, $\tilde{y}$), \ size constants}

\sld{Bayes's Rule for the Posterior}
% 
\begin{itemize}
\item Suppose the data $y$ is fixed (i.e., observed).  Then
  % 
  \vspace*{2pt}
  \begin{eqnarray*}
    p(\theta|y) 
    \hspace*{8pt} = \hspace*{8pt} \frac{p(y,\theta)}{p(y)}
    & = & \frac{p(y|\theta) \, p(\theta)}{p(y)}
    \\[6pt]
    & = & \frac{p(y|\theta) \, p(\theta)}{\int p(y,\theta) \ d\theta}
    \\[6pt]
    & = & \frac{p(y|\theta) \, p(\theta)}{\int p(y|\theta) \, p(\theta) \ d\theta}
    \\[6pt]
    & \propto & p(y|\theta) \, p(\theta) \ \ = \ \ p(y,\theta)
  \end{eqnarray*}
\item Posterior proportional to likelihood times prior (i.e., joint)
\end{itemize}

\sld{``Naive Bayes'' Four Ways}
\begin{itemize}
%
\item Joint Distribution \ $p(\pi,\phi,z,w)$, defined by:
\vspace*{-3pt}
\begin{itemize}\small
\item $\pi \sim \mbox{\sf Dirichlet}(\alpha)$    \hfill (topic prevalence)
\item $\phi_k \sim \mbox{\sf Dirichlet}(\beta)$  \hfill (word
  prevalence in topic $k$)
\item $z_d \sim \mbox{\sf Categorical}(\pi)$     \hfill (topic for doc $d$)
\item $w_{d,n} \sim \mbox{\sf Categorical}(\phi_{z_d})$  \hfill (word
  $n$ in doc $d$)
\end{itemize}
%
\item Inference Problem \ $p(\tilde{z},\phi,\pi |  w, z, \tilde{w})$
\vspace*{-3pt}
\begin{itemize}\small
\item fully supervised learning: \ $p(\pi,\phi | w, z)$
\item semi-supervised learning: \ $p(\pi, \phi | w, z,
  \tilde{w})$
\item clustering: \ $p(\tilde{z} | \tilde{w})$
\item prediction: \ $p(\tilde{z} | w, z, \tilde{w})$
\end{itemize}
\end{itemize}



\sld{Monte Carlo Methods}
\begin{itemize}
\item For integrals that are impossible to solve analytically
\item But for which sampling and evaluation is tractable
\item Compute plug-in estimates of statistics based on
  randomly generated variates (e.g., means, variances,
  quantiles/intervals, comparisons)
\item Accuracy with $M$ (independent) samples proportional to
  \[
  \frac{1}{\sqrt{M}}
  \]
  e.g., 100 times more samples per decimal place!
\end{itemize}
\vfill\hfill
{\small (Metropolis and Ulam 1949)}


\sld{Monte Carlo Example}
\begin{itemize}
\item Posterior expectation of $\theta$:
  \[
  \mathbb{E}[\theta|y] = \int \theta \ p(\theta|y) \ d\theta.
  \]
\item Bayesian estimate minimizing expected square error: 
  \[
  \hat{\theta} 
  = \arg\min_{\theta'}
  \mathbb{E}[(\theta - \theta')^2|y]
  = \mathbb{E}[\theta|y] 
  \]
\item Generate samples $\theta^{(1)}, \theta^{(2)}, \ldots,
  \theta^{(M)}$ drawn from $p(\theta|y)$
\item Monte Carlo Estimator plugs in average for expectation:
  \[
  \mathbb{E}[\theta|y] \approx \frac{1}{M} \sum_{m=1}^M \theta^{(m)}
  \]
\end{itemize}

\sld{Monte Carlo Example II}
\begin{itemize}
\item Bayesian alternative to frequentist hypothesis testing
\item Use probability to summarize results
\item Bayesian comparison: probability $\theta_1 > \theta_2$ given
  data $y$?
  \begin{eqnarray*}
    \mbox{Pr}[\theta_1 > \theta_2|y] 
    & = & 
    \int \int 
    \mathbb{I}(\theta_1 > \theta_2) \ p(\theta_1|y) \ p(\theta_2|y)  
    \ d\theta_1 \ d\theta_2
    \\
    & \approx & 
    \frac{1}{M} \sum_{m=1}^M \mathbb{I}(\theta_1^{(m)} > \theta_2^{(m)})
  \end{eqnarray*}
  % 
\item (Bayesian hierarchical model ``adjusts'' for multiple comparisons)
\end{itemize}

\sld{Markov Chain Monte Carlo}
\begin{itemize}
\item When sampling independently from $p(\theta|y)$ impossible
\item $\theta^{(m)}$ drawn via a Markov chain $p(\theta^{(m)}|y,\theta^{(m-1)})$
\item Require MCMC marginal $p(\theta^{(m)}|y)$ equal to true
  posterior marginal
\item Leads to auto-correlation in samples
  $\theta^{(1)},\ldots, \theta^{(m)}$
\item Effective sample size $N_{\mbox{\tiny eff}}$ divides out
  autocorrelation (must be estimated)
\item Estimation accuracy proportional to $1 / \sqrt{N_{\mbox{\tiny eff}}}$
\end{itemize}

\sld{Gibbs Sampling}
\begin{itemize}
\item Samples a parameter given data and other parameters
\item Requires conditional posterior $p(\theta_n|y,\theta_{-n})$
\item Conditional posterior easy in directed graphical model
\item Requires general unidimensional sampler for non-conjugacy

  \begin{itemize}
  \item JAGS uses slice sampler
  \item BUGS uses adaptive rejection sampler
  \end{itemize}
\item Conditional sampling and general unidimensional sampler 
  can both lead to slow convergence and mixing
\end{itemize}
\vfill\hfill
{\small (Geman and Geman 1984)}

\sld{Metropolis-Hastings Sampling}
\begin{itemize}
\item Proposes new point by changing all parameters randomly
\item Computes accept probability of new point based
  on ratio of new to old log probability (and proposal density)
\item Only requires evaluation of $p(\theta|y)$
\item Requires good proposal mechanism to be effective
\item Acceptance requires small changes in log probability
\item But small step sizes lead to random walks and slow convergence
  and mixing
\end{itemize}
\vfill\hfill
{\small (Metropolis et al. 1953; Hastings 1970)}


\end{document}


